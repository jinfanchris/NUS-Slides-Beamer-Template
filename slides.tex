% Copyright 2016 by Wang Kunzhen <wangkunzhen1993@gmail.com>.
% Modified 2025 by Jinfan LIU <jinfan@u.nus.edu>.
%
% This is a latex template adapted from Till Tantau's Beamer template.
% It adds theme customizations for the convenience of users from the
% National University of Singapore. 
% 
% In principle, this file can be redistributed and/or modified under
% the terms of the GNU Public License, version 2.
%
% However, this file is supposed to be a template to be modified
% for your own needs. For this reason, if you use this file as a
% template and not specifically distribute it as part of a another
% package/program, I grant the extra permission to freely copy and
% modify this file as you see fit and even to delete this copyright
% notice.

\documentclass[xcolor=dvipsnames]{beamer}
\usepackage{nus}

\title{Title}
\subtitle{SubTitle}
\author{Author's Name}

\institute[National University of Singapore] % (optional, but mostly needed)
{
    School of Computing\\
    National University of Singapore
}


\logo{\includegraphics[height=0.9cm]{nus-logo.jpg}}
\date{}


\begin{document}


\begin{frame}

    \titlepage
\end{frame}

\setbeamercolor{normal text}{fg=nus-blue,bg=nus-white}


\begin{frame}{Outline}
    \tableofcontents
\end{frame}


\section{First Main Section}

\subsection{First Subsection}
\begin{frame}{First Slide Title}{Optional Subtitle}
    \begin{itemize}
        \item {
                My first point.
            }
        \item {
                My second point.
            }
    \end{itemize}
\end{frame}

\subsection{Second Subsection}
% You can reveal the parts of a slide one at a time
% with the \pause command:
\begin{frame}{Second Slide Title}
    \begin{itemize}
        \item {
                First item.
                \pause % The slide will pause after showing the first item
            }
            % You can also specify when the content should appear
            % by using <n->:
        \item<3-> {
                Third item.
            }
            % or you can use the \uncover command to reveal general
            % content (not just \stems):
        \item<5-> {
                Fifth item. \uncover<6->{Extra text in the fifth item.}
            }
    \end{itemize}
\end{frame}

\section{Second Main Section}

\subsection{Second Subsection}
\begin{frame}{Main Theorem}
    \begin{theorem}
        Theorem Statements. Example for citation \cite{Author1990}.
    \end{theorem}

    \begin{proof}
        Proof of the theorem goes here.
    \end{proof}
\end{frame}

% Placing a * after \section means it will not show in the
% outline or table of contents.
\section*{Summary}

\begin{frame}{Summary}
    \begin{itemize}
        \item
            The \alert{first main message} of your talk in one or two lines.
        \item
            The \alert{second main message} of your talk in one or two lines.
        \item
            Perhaps a \alert{third message}, but not more than that.
    \end{itemize}

    \begin{itemize}
        \item
            Outlook
            \begin{itemize}
                \item
                    Something you haven't solved.
                \item
                    Something else you haven't solved.
            \end{itemize}
    \end{itemize}
\end{frame}

\section*{Bibliography}
\begin{frame}[allowframebreaks]{References}
    \footnotesize
    \bibliographystyle{ieeetr}
    \bibliography{references}
\end{frame}
\end{document}
